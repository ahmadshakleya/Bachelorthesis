\documentclass[../main.tex]{subfiles}

\setstretch{0.1} % 

\begin{document}
\chapter{Conclusion}
%The application successfully demonstrated its capability in both simulating and reconstructing complex physical phenomena, supported by a user-friendly and powerful graphical interface. Future work will focus on expanding the range of simulations and incorporating machine learning techniques for automated image analysis and feature recognition.

%A significant contribution of this project is the development of a graphical user interface (GUI) that amalgamates these simulation tools into a user-friendly platform. This GUI enables researchers and engineers to model, visualize, and analyze the behavior of optical flats dynamically, supporting iterative testing and parameter adjustments. This development significantly enhances the efficiency of research and development processes in optical engineering.
% \subsection{Innovations and Contributions to the Field}
% The synthesis of simulation tools within an all-encompassing GUI, paired with the advanced data processing capabilities such as the FT method for phase recovery, constitutes a significant advancement in the field of optical engineering. This project not only refines the accuracy and operational efficiency of optical flat simulations but also extends these sophisticated techniques to practical, real-world applications.

% These developments are poised to enhance the design and manufacturing processes of optical components, impacting a variety of sectors including microscopy, lens manufacturing, and large-scale optical testing, where precision and efficiency are critical.

\end{document}