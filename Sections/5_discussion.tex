\documentclass[../main.tex]{subfiles}

\setstretch{0.1} % 

\begin{document}
%\chapter{Discussion}
{\let\clearpage\relax\chapter{Discussion}}


%This method of using planes separated by specific intervals corresponding to the light wavelength proved effective in simulating the necessary conditions to study optical flats. The Python environment, complemented by a user-friendly GUI, enhanced the flexibility and accessibility of the simulation, making it a robust tool for both educational and research applications in optical measurements.

%The representativeness of the simulations and reconstructions was not evaluated due to lack of time. This could be done by comparing the generated models and extracted data against known standards and real samples. This evaluation indicates that the methods are designed to minimize errors and enhance the accuracy of the simulations. Iterative testing and refining of the simulation parameters should show that they closely align with actual measurements.

% \section{Research Goals}
% As highlighted in the Introduction, the primary objective of this research was to develop a robust simulation and reconstruction framework capable of accurately modeling and analyzing complex optical phenomena and surface topographies. The simulation results demonstrated that the Python-based modules could successfully replicate realistic optical interactions and interference patterns. This aligns well with the project's aim to provide a tool for enhancing the precision of optical measurements, as theorized in our initial research questions.

% The ability to reconstruct accurate phase maps and 3D surface topographies from both simulated and real-world data underscores the effectiveness of the Fourier Transform methods and phase unwrapping techniques employed. These results directly correlate with the methodological approaches discussed in the Methods and Materials section, where the integration of advanced image processing algorithms was anticipated to significantly improve the accuracy and reliability of optical data analysis.

% \section{Representativeness and Limitations of 3D Reconstruction}
% Despite the successes, the 3D reconstruction process was noted as not fully representative in some aspects. The limitations primarily stemmed from the inherent challenges in accurately capturing the nuances of surface topographies solely from optical measurements. This was partly due to the resolution limits of the imaging techniques used, which could not always capture the full depth and detail required for a fully representative 3D model. Besides that, we used a binary representation for the interference fringes, in which there are sharp transistions between dark fringes and bright fringes. This causes the 3D representation to show sharp transistions.

% Moreover, the lack of comprehensive evaluation against known standards or real samples, as noted in the Results section, prevented a thorough validation of the representativeness of the reconstructed models. In scenarios where high fidelity and precision are critical, such as in quality control or materials science research, these limitations could impact the practical applicability of the reconstructions.

% \section{Future Directions}
% To address these limitations, future research could focus on integrating multi-modal imaging techniques that combine optical measurements with other forms of data acquisition, such as electron microscopy or CT scanning, to enhance the depth and accuracy of the 3D reconstructions. Also, more interference fringes could be simulated using not only intersections with planes at half the wavelength used, but also the in between values.

% Additionally, implementing a more rigorous validation framework that includes comparison against established benchmarks and real-world objects would be crucial in enhancing the representativeness and utility of the reconstructed models.

\end{document}