\documentclass[../main.tex]{subfiles}
\begin{document}

\chapter{Research Questions}

\section{Overview of Research Questions}
This chapter outlines several key research questions that guide the investigation into the properties, simulation, and applications of optical flats. These questions are designed to explore both fundamental aspects and practical implementations.

\section{What is an Optical Flat?}
Understanding the fundamental nature and characteristics of optical flats is crucial. This involves examining their material composition, surface quality, and the precision standards they meet in optical testing and engineering applications.

\section{How Can We Simulate an Optical Flat?}
Simulation of optical flats involves creating virtual models that can predict and visualize the behavior of light when interacting with flat or nearly flat surfaces. This question explores the methods and technologies, such as computational modeling and software tools, used to simulate the optical phenomena associated with optical flats.

\section{How Can We Extract 3D Information from the Measurements?}
Extracting three-dimensional (3D) information from the interference patterns generated by optical flats is vital for assessing the topography of surfaces under examination. This includes methodologies for interpreting fringe patterns and converting them into quantifiable 3D surface data.

\section{Is it Representative Enough?}
This question addresses the representativeness of optical flats in various applications. It involves evaluating whether optical flats provide sufficient accuracy and resolution for the applications they are used in, from fundamental research to industrial quality control.

\section{Simulation}
Further detailing the simulation aspect, this section should delve into specific algorithms, computational techniques, and the challenges faced in accurately simulating the interaction of light with various surface types using optical flats.

\section{Reconstruction}
Reconstruction focuses on the process of translating fringe patterns observed with optical flats into detailed surface maps. This includes discussing the tools and algorithms used for digital image processing and data interpretation to reconstruct accurate surface models from measurement data.

\end{document}
