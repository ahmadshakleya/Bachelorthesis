\documentclass[../main.tex]{subfiles}

\setstretch{0.1} % voor de interlinie

\begin{document}

%\chapter{}
{\let\clearpage\relax\chapter{Research Questions}}
This chapter outlines several key research questions that guide our investigation into the properties, simulation, and applications of optical flats. We begin by exploring the fundamental nature and characteristics of optical flats. This includes examining their material composition, surface quality, and the precision standards they meet in optical testing and engineering applications. Key aspects include the physical characteristics of optical flats, their operational principles, and the ongoing research in this area.

\textbf{How can we simulate an optical flat?} Our investigation next turns to the simulation of optical flats, focusing on creating virtual models that can predict and visualize the behavior of light when interacting with these surfaces. We discuss the potential of using a ray-tracing approach or a plane wavefront model, and weigh the advantages and disadvantages of each simulation method.

\textbf{How can we extract 3D surface shape from the measurements?} Another critical aspect of our research is extracting three-dimensional (3D) information from the interference patterns generated by optical flats. We review various methods from the literature for interpreting these fringe patterns and converting them into quantifiable 3D surface data, considering the integration of these methods with images generated from our simulations.

\textbf{Are the simulation and reconstruction representative enough?} Finally, we assess the representativeness of our simulations and reconstructions in various applications, examining how much the reconstructions deviate from the real models and exploring ways to enhance these reconstructions for better accuracy and practical utility.

\end{document}
