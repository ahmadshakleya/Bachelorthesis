\documentclass[../main.tex]{subfiles}
\begin{document}

\chapter{Research Questions}

\section{Overview of Research Questions}
This chapter outlines several key research questions that guide the investigation into the properties, simulation, and applications of optical flats. These questions are designed to explore both fundamental aspects and practical implementations.

\section{What is an Optical Flat?}
Understanding the fundamental nature and characteristics of optical flats is crucial. This involves examining their material composition, surface quality, and the precision standards they meet in optical testing and engineering applications.
\begin{itemize}
    \item What are the physical characteristics?
    \begin{itemize}
        \item What is it composed of?
        \item How do we define its surface quality?
        \item What are applications w.r.t. those characteristics?
    \end{itemize}
    \item How does it work?
    \item What is the current research being done arround optical flats?
\end{itemize}

\section{How Can We Simulate an Optical Flat?}
Simulation of optical flats involves creating virtual models that can predict and visualize the behavior of light when interacting with flat or nearly flat surfaces. This question explores the methods and technologies, such as computational modeling and software tools, used to simulate the optical phenomena associated with optical flats.
\begin{itemize}
    \item Is it possible to simulate using a ray-tracing approach?
    \item Is it possible to simulate using a plane wavefront model? 
    \item What are the advantages of using a simulation?
    \item What are the disadvantages of using a simualtion?
\end{itemize}

\section{How Can We Extract 3D Information from the Measurements?}
Extracting three-dimensional (3D) information from the interference patterns generated by optical flats is vital for assessing the topography of surfaces under examination. This includes methodologies for interpreting fringe patterns and converting them into quantifiable 3D surface data.
\begin{itemize}
    \item How do we mathematically extract useful data out of the measurements?
    \item How do we extract useful data out of it using a signal processing mindset?
    \item Can we combine those 2 approaches?
\end{itemize}

\section{Is it Representative Enough?}
This question addresses the representativeness of optical flats in various applications. It involves evaluating whether optical flats provide sufficient accuracy and resolution for the applications they are used in, from fundamental research to industrial quality control.
\begin{itemize}
    \item How much does the reconstruction deviate from the real model?
    \begin{itemize}
        \item How can we enhance this reconstruction?
    \end{itemize}
\end{itemize}

\end{document}
