\tikzstyle{startstop} = [rectangle, rounded corners, minimum width=1cm, minimum height=1cm,text centered, draw=black, fill=red!30, font=\scriptsize]
\tikzstyle{io} = [trapezium, trapezium left angle=70, trapezium right angle=110, minimum width=3cm, minimum height=1cm, text centered, draw=black, fill=blue!30]
\tikzstyle{process} = [rectangle, minimum width=3cm, minimum height=1cm, text centered, draw=black, fill=orange!30]
\tikzstyle{decision} = [diamond, minimum width=3cm, minimum height=1cm, text centered, draw=black, fill=green!30]
\tikzstyle{arrow} = [thick,->,>=stealth]

\begin{figure}[H]
\centering
\begin{tikzpicture}[node distance=2cm]
\node (start) [startstop] {Start};
\node (in1) [io, below of=start] {Input: shape1, shape2};
\node (pro1) [process, below of=in1] {Get surface points for both shapes};
\node (dec1) [decision, below of=pro1, yshift=-1.5cm] {Point within surface bounds?};
\node (pro2a) [process, below of=dec1, yshift=-1.5cm] {Find closest indices in surface grid};
\node (dec2) [decision, below of=pro2a, yshift=-1cm] {Z-value intersection?};
\node (pro2b) [process, below of=dec2, yshift=-1cm] {Record intersection point};
\node (out1) [io, below of=pro2b] {Output: intersection points};
\node (stop) [startstop, below of=out1] {Stop};

\draw [arrow] (start) -- (in1);
\draw [arrow] (in1) -- (pro1);
\draw [arrow] (pro1) -- (dec1);
\draw [arrow] (dec1) -- node[anchor=east] {yes} (pro2a);
\draw [arrow] (dec1.east) -- ++(3,0) node[anchor=west] {no} |- (pro1.east);
\draw [arrow] (pro2a) -- (dec2);
\draw [arrow] (dec2) -- node[anchor=east] {yes} (pro2b);
\draw [arrow] (dec2.east) -- ++(3,0) node[anchor=west] {no} |- (pro2a.east);
\draw [arrow] (pro2b) -- (out1);
\draw [arrow] (out1) -- (stop);
\end{tikzpicture}
\caption{Flowchart of the Intersection Calculation Process}
\label{fig:flowchart1}
\end{figure}


\begin{figure}[H]
    \centering
    \begin{tikzpicture}[scale=1, transform shape, node distance=1cm, auto]
    % Styles for flowchart elements
    \tikzstyle{startstop} = [rectangle, rounded corners, minimum width=0.5cm, minimum height=0.5cm,text centered, draw=black, fill=red!30, font=\scriptsize]
    \tikzstyle{io} = [trapezium, trapezium left angle=70, trapezium right angle=110, minimum width=0.5cm, minimum height=0.5cm, text centered, draw=black, fill=blue!30, font=\scriptsize]
    \tikzstyle{process} = [rectangle, minimum width=0.5cm, minimum height=0.5cm, text centered, draw=black, fill=orange!30, font=\scriptsize]
    \tikzstyle{decision} = [diamond, minimum width=0.5cm, minimum height=0.2cm, text centered, draw=black, fill=green!30, font=\scriptsize]
    \tikzstyle{arrow} = [thick,->,>=stealth]
    
    % Nodes
    \node (start) [startstop, xshift=-4cm] {Start};
    \node (in1) [io, below of=start] {Input: shape1, shape2};
    \node (pro1) [process, below of=in1] {Get surface points for both shapes};

    \node (dec1) [decision, right of=start, xshift=4cm] {Point within surface bounds?};
    \node (pro2a) [process, below of=dec1] {Find closest indices in surface grid};
    \node (dec2) [decision, below of=pro2a] {Z-value intersection?};

    \node (pro2b) [process, right of=dec1, xshift=4cm] {Record intersection point};
    \node (out1) [io, below of=pro2b] {Output: intersection points};
    \node (stop) [startstop, below of=out1] {Stop};
    
    % Arrows
    \draw [arrow] (start) -- (in1);
    \draw [arrow] (in1) -- (pro1);
    \draw [arrow] (pro1.north east) -- ++(0,2) node[anchor=east] {} -- (dec1.west);
    \draw [arrow] (dec1) -- node[anchor=east] {yes} (pro2a);
    \draw [arrow] (dec1.west) -- ++(-1,0) node[anchor=west] {no} -- (pro1.east);
    \draw [arrow] (pro2a) -- (dec2);
    \draw [arrow] (dec2) -- node[anchor=east] {yes} (pro2b);
    \draw [arrow] (dec2.east) -- ++(3,0) node[anchor=west] {no} |- (pro2a.east);
    \draw [arrow] (pro2b) -- (out1);
    \draw [arrow] (out1) -- (stop);
    
    \end{tikzpicture}
    \caption{Flowchart of the Intersection Calculation Process}
    \label{fig:flowchart1}
    \end{figure}
